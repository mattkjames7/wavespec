\documentclass[a4paper,10pt]{article}
%\documentclass[a4paper,10pt]{scrartcl}

\usepackage[utf8]{inputenc}
\usepackage{lineno}
\usepackage{amsmath}
\usepackage{amsfonts}
\usepackage{amssymb}
\usepackage{graphicx}
\usepackage{times}
\usepackage[T1]{fontenc}  % to get the appropriate fonts for icelandic thorn
\usepackage{textcomp}
\usepackage{natbib}
\usepackage{placeins}
\usepackage{listings}
\usepackage{xcolor}

\definecolor{codegreen}{rgb}{0,0.6,0}
\definecolor{codered}{rgb}{1.0,0.0,0.0}
\definecolor{codeblue}{rgb}{0.0,0,1.0}
\definecolor{codeyellow}{rgb}{1.0,0.7,0.0}
\definecolor{backcolour}{rgb}{0.95,0.95,0.92}

\lstdefinestyle{mystyle}{
	backgroundcolor=\color{backcolour},   
	commentstyle=\color{codered},
	keywordstyle=\color{codeblue},
	numberstyle=\tiny\color{codegreen},
	stringstyle=\color{codeyellow},
	basicstyle=\ttfamily\footnotesize,
	breakatwhitespace=false,         
	breaklines=true,                 
	captionpos=b,                    
	keepspaces=true,                 
	numbers=left,                    
	numbersep=5pt,                  
	showspaces=false,                
	showstringspaces=false,
	showtabs=false,                  
	tabsize=2
}

\lstset{style=mystyle}
\DeclareMathOperator{\atantwo}{arctan2}

\title{Lomb-Scargle Algorithm Documentation}
\author{Matt James}
\date{}

\pdfinfo{%
  /Title    ()
  /Author   ()
  /Creator  ()
  /Producer ()
  /Subject  ()
  /Keywords ()
}

\begin{document}
\maketitle
	
\section{Description of Lomb-Scargle Periodogram}

Everything in this document is pieced together from \citet{Lomb1976,Scargle1983,Hocke1998}. This algorithm will calculate the power, $P$, amplitude, $A$, and phase, $\phi$, of a wave with a given frequency, $f$, within an irregularly sampled time series of length $n$ samples.

Let us consider sinusoidal waves of the form 
\begin{equation}
	y_f(t_i) = a \cos{\omega(t_i - \tau)} + b \sin{\omega(t_i - \tau)}, \label{EqWave}
\end{equation}
where $\omega = 2 \pi f$ and $1 \le i \le n$.

$\tau$ is defined in \citet{Scargle1983} as
\begin{equation}
	\tan{2 \omega \tau} = \frac{\sum_{i=1}^{n} \sin{2 \omega t_i}}{\sum_{i=1}^{n} \cos{2 \omega t_i}},
\end{equation}
which can easily be rewritten as
\begin{equation}
	\tau = \frac{\atantwo{\left( \sum_{i=1}^{n} \sin{2 \omega t_i}, \sum_{i=1}^{n} \cos{2 \omega t_i} \right)}}{2 \omega}
\end{equation}

The constants $a$ and $b$ are defined as
\begin{equation}
	a = \frac{\sqrt{\frac{2}{n}}\sum_{i=1}^{n}y_i \cos{\omega (t+i - \tau)}}{\left( \sum_{i=1}^{n} \cos^2{\omega (t+i - \tau)} \right)^{\frac{1}{2}}}
\end{equation}
and
\begin{equation}
b = \frac{\sqrt{\frac{2}{n}}\sum_{i=1}^{n}y_i \sin{\omega (t+i - \tau)}}{\left( \sum_{i=1}^{n} \sin^2{\omega (t+i - \tau)} \right)^{\frac{1}{2}}}.
\end{equation}

The periodogram is calculated using 
\begin{equation}
	P(\omega) = \frac{1}{2\sigma^2}\frac{n}{2} (a^2 + b^2),
\end{equation}
where $\sigma = \frac{1}{n-1}\sum_{i=1}^{n} y_i^2$ is the variance (assuming that the mean has been subtracted from the data already).

The amplitude is calculated using 
\begin{equation}
	A(\omega) = \sqrt{\frac{4\sigma^2}{n}P(\omega)}
\end{equation}
or, equivalently,
\begin{equation}
	A(\omega) = \sqrt{a^2 + b^2}.
\end{equation}

Equation \ref{EqWave} can be expressed differently:
\begin{equation}
	y_f(t_i) = A(\omega) \cos{\left[\omega (t_i - \tau) + \phi \right]},
\end{equation}
where the wave phase is given by,
\begin{equation}
	\phi = -\atantwo(b,a).
\end{equation}

\section{Example Python Code}
The code used in this module is based on the following Python code.
						
						
\begin{lstlisting}[language=Python]
def LombScargle(t,x,f):
	'''
	Calculates the Lomb-Scargle periodogram using the method defined in
	Hocke 1998. This method assumes that any mean is removed from the
	data.
	'''
	
	#preformat the input variables
	t = np.array([t],dtype='float64').flatten()
	x = np.array([x],dtype='float64').flatten()
	f = np.array([f],dtype='float64').flatten()
	
	#get array sizes
	nf = np.size(f)
	n = np.size(t)
	
	#create output arrays (Power, amplitude, phase, a, b)
	P = np.zeros((nf,),dtype='float64')
	A = np.zeros((nf,),dtype='float64')
	phi = np.zeros((nf,),dtype='float64')
	a = np.zeros((nf,),dtype='float64')
	b = np.zeros((nf,),dtype='float64')
	
	#convert f to omega
	w = 2*np.pi*f
	
	#calculate variance (sigma**2)
	o2 = np.sum(x**2)/(n-1)
	
	#loop through each frequency
	for i in range(0,nf):
		#calculate sums to get Tau
		ss2w = np.sum(np.sin(2*w[i]*t))
		sc2w = np.sum(np.cos(2*w[i]*t))
	
		#calculate Tau
		Tau = np.arctan2(ss2w,sc2w)/(2*w[i])
		
		#calculate w(t - Tau)
		wtT = w[i]*(t - Tau)
		
		#calculate some more sums
		syc = np.sum(x*np.cos(wtT))
		sys = np.sum(x*np.sin(wtT))
		sc2 = np.sum(np.cos(wtT)**2)
		ss2 = np.sum(np.sin(wtT)**2)
	
		#get a  and b
		rt2n = np.sqrt(2/n)
		a[i] = rt2n*syc/np.sqrt(sc2)
		b[i] = rt2n*sys/np.sqrt(ss2)
	
	#calculate the periodogram
	a2b2 = a**2 + b**2
	P[:] = (n/(4*o2))*a2b2
	
	#calculate amplitude
	A[:] = np.sqrt(a2b2)
	
	#calculate phase
	phi[:] = -np.arctan2(b,a)
	
	return P,A,phi,a,b
\end{lstlisting}
\bibliographystyle{agu08}
\bibliography{references}
\end{document}
